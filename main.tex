\PassOptionsToPackage{table}{xcolor}
\documentclass[justified,openany,nofonts]{tufte-book}
\usepackage{mooculus}
\usepackage[utf8]{inputenc}
\usepackage{ngerman}
%\usepackage{showkeys} %% Useful for debugging

\setcounter{secnumdepth}{2}
\renewcommand{\figurename}{Abbildung}



% Prints the month name (e.g., January) and the year (e.g., 2008)
\newcommand{\monthyear}{%
  \ifcase\month\or January\or February\or March\or April\or May\or June\or
  July\or August\or September\or October\or November\or
  December\fi\space\number\year
}

% Generates the index
\usepackage{makeidx}
\makeindex

\newcommand{\xrefn}[1]{\ref{#1}}
\renewcommand{\tablename}{Tabelle}


\newenvironment{lemma}{\subsection*{Lemma}}{}
\newenvironment{remark}[1]{\subsection*{Remark: #1}}{}


%\def\exam{\null}
\def\pagerdef{\null}

\def\dfont{\bf}
\def\em{\it}           % for emphasis

\newcommand{\ds}{\displaystyle}

\let\ssk\smallskip \let\msk\medskip \let\bsk\bigskip

\usepackage{multicol}
\def\twocol{\begin{multicols}{2}}
\def\endtwocol{\end{multicols}}

\title{Integralrechnung}
%\author{Jim Fowler and Bart Snapp}
\publisher{Zuletzt bearbeitet: \today.}
%\newcommand{\blankpage}{\newpage\hbox{}\thispagestyle{empty}\newpage}

%% % Prints an epigraph and speaker in sans serif, all-caps type.
%% \newcommand{\openepigraph}[2]{%
%%   %\sffamily\fontsize{14}{16}\selectfont
%%   \begin{fullwidth}
%%   \sffamily\large
%%   \begin{doublespace}
%%   \noindent\allcaps{#1}\\% epigraph
%%   \noindent\allcaps{#2}% author
%%   \end{doublespace}
%%   \end{fullwidth}
%% }




\begin{document}
\maketitle

% v.4 copyright page


\begin{fullwidth}
~\vfill
\thispagestyle{empty}
\setlength{\parindent}{0pt}
\setlength{\parskip}{\baselineskip}
Matthias Heimberg

Gymnasium Oberaargau \url{gymo.ch}



\end{fullwidth}


\tableofcontents


%% \renewcommand{\listtheoremname}{List of Main Theorems}
%% \setcounter{tocdepth}{1}
%% \listoftheorems[numwidth=4em,ignoreall,show={mainTheorem}]



\chapter*{Die wichtigsten Sätze}% uses ntheorem
\theoremlisttype{opt}
\listtheorems{mainTheorem}



\chapter*{Wie Sie ein Mathematikskript lesen}

Ein Mathematikskript können Sie \textbf{nicht} wie einen Roman lesen! Um Mathematik zu lesen benötigen Sie:
\begin{enumerate}
\item Einen Stift.
\item Viel leeres Papier.
\item Den Willen, Dinge niederzuschreiben.
\end{enumerate}
Wenn Sie Mathematik lesen, sollten Sie gleichzeitig diese Dinge ver\textbf{arbeiten}. Das bedeutet: Jeden Ausdruck \textbf{aufschreiben}, jeden Graphen \textbf{skizzieren}, darüber \textbf{nachdenken} was Sie machen. Arbeiten Sie die Beispiele selbst durch und ergänzen Sie die Details. Dies ist keine simple Aufgabe, es ist \textbf{harte} Arbeit! Mathematik ist keine passive Reise, es passiert nicht von selbst, Sie müssen selbst die Mathematik machen!



\setcounter{chapter}{0}

%%
% Start the main matter (normal chapters)
%\mainmatter
%\tikzexternaldisable
%\input Funktionen
%\input Grenzwerte
%\input infinityAndContinuity
%\input GrundlagenDerAbleitung
%\input curveSketching
%\input productRuleAndQuotientRule
%\input Kettenregel
%\input AbleitungTrigFkt
%\input AnwendungenDiff
%\input optimization
%\input linearApproximation
\input 1Stammfunktionen
\input 2Integrale
\input 3Hauptsatzdiffintegral
\input 4Methoden
\input 5Anwendungen




%\bibliography{sample-handout}
%\bibliographystyle{plainnat}

%\finalizeanswers
%\chapter*{Answers to Exercises}
%\small
%\addcontentsline{toc}{chapter}{Answers to Exercises}
%\IfFileExists{answers.tex}{\input{answers}}
%\normalsize
%\backmatter
%\addcontentsline{toc}{chapter}{Index}
\printindex


\end{document}



%sagemathcloud={"zoom_width":95}