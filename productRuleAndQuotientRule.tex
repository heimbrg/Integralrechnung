\chapter{Produktregel und Quotientenregel der Ableitung}

\section{Die Produktregel}


Betrachten Sie das Produkt zweier Funktionen, zum Beispiel
\[
f(x)\cdot g(x)
\]
mit $f(x)=x^2+1$ und $g(x)=x^3-3x$. Intuitiv könnte man meinen, dass die Ableitung von $f(x)g(x)$ das Produkt der Ableitungen der beiden Funktionen ist:
\begin{align*}
f'(x)g'(x) &= (2x)(3x^2-3)\\
&= 6x^3-6x.
\end{align*}
Ist das korrekt? Wir können dies überprüfen, in dem wir das Produkt von $f(x)$
und $g(x)$ ausrechnen und die Ableitung davon machen, so wie wir dies aus Kapitel 2 kennen:
\begin{align*}
f(x)g(x) &= (x^2+1)(x^3-3x)\\
&=x^5-3x^3+x^3-3x\\
&=x^5-2x^3-3x.
\end{align*} 
Also
\[
\ddx f(x) g(x) = 5x^4-6x^2-3, 
\]
damit sehen wir, dass
\[
\ddx f(x) g(x) \ne  f'(x)g'(x).
\]
Also ist die Ableitung von $f(x)g(x)$ \textbf{nicht} so einfach wie
$f'(x)g'(x)$. Wir haben jedoch trotzdem eine Regel für solche Situationen:
\begin{mainTheorem}[Produktregel]\index{Ableitungsregeln! Produkt}\index{Produktregel}\label{theorem:product-rule}
Sind $f(x)$ und $g(x)$ differenzierbar, dann gilt
\[
\ddx f(x)g(x) = f(x)g'(x)+f'(x)g(x).
\]
\end{mainTheorem}

\begin{marginfigure}
\begin{tikzpicture}
	\begin{axis}[
            clip=false,
            domain=1.5:6, 
            ytickmin=1,ytickmax=0,
            xtick={4},ytickmin=1,ytickmax=0,
            xticklabels={$a$},
            ymin=0, ymax=6,
            xlabel=$x$, ylabel=$y$,
            axis lines=center,
            every axis y label/.style={at=(current axis.above origin),anchor=south},
            every axis x label/.style={at=(current axis.right of origin),anchor=west},
            axis on top,
          ]          
          %\addplot [dashed, textColor] plot coordinates {(4,0) (4,3.08)};
          %\node at (axis cs:4,0) [anchor=north] {$x$};

          \addplot [penColor5,very thick] plot coordinates {(5,1.4) (5,1.9)};
          \addplot [dashed, very thick, textColor] plot coordinates {(4,1.4) (5,1.4)};

          \addplot [penColor4,very thick] plot coordinates {(5,2.2) (5,2.6)};
          \addplot [dashed, very thick, textColor] plot coordinates {(4,2.2) (5,2.2)};

          \addplot [very thick, penColor5!50!penColor2] plot coordinates {(5,3.08) (5,4.18)};
          \addplot [very thick, penColor4!50!penColor] plot coordinates {(5,4.18) (5,4.74)};
          \addplot [dashed, very thick, textColor] plot coordinates {(4,3.08) (5,3.08)};
        
          \addplot [very thick,penColor,smooth] {-.6+.5*x};
          \addplot [very thick,penColor2,smooth] {.6+.4*x};         
          \addplot [very thick,penColor3,smooth,domain=1:5.5] {-.36+.06*x+.2*x^2};      
          \addplot [penColor3!70!background,smooth, domain=1.6:5.6] {-3.56+1.66*x};      
          
          \addplot [dashed, textColor] plot coordinates {(4,0) (4,3.08)};

          \node at (axis cs:3.5,1.1) [anchor=north,penColor] {$f(x)$};
          \node at (axis cs:2,1.95) [anchor=north,penColor2] {$g(x)$};
          \node at (axis cs:4.7,4.3) [anchor=east,penColor3] {$f(x)g(x)$};

          \node at (axis cs:5,3.91) [anchor=west] {${\color{penColor}f(a)}{\color{penColor4}g'(a)h}+{\color{penColor5}f'(a)h}{\color{penColor2}g(a)} + {\color{penColor5}f'(a)h}{\color{penColor4}g'(a)h}$};
          \node at (axis cs:5,1.65) [anchor=west,penColor5] {$f'(a)h$};
          \node at (axis cs:5,2.4) [anchor=west,penColor4] {$g'(a)h$};
          \node at (axis cs:4.5,1.5) [anchor=north] {$\underbrace{\hspace{.40in}}_{h}$};

          \addplot[color=penColor,fill=penColor,only marks,mark=*] coordinates{(4,1.4)};  %% closed hole          
          \addplot[color=penColor2,fill=penColor2,only marks,mark=*] coordinates{(4,2.2)};  %% closed hole          
          \addplot[color=penColor3,fill=penColor3,only marks,mark=*] coordinates{(4,3.08)};  %% closed hole          
        \end{axis}
\end{tikzpicture}
\caption[Eine geometrische Interpretation der Produktregel.]{Eine geometrische Interpretation der Produktregel. Da jeder Punkt von $f(x)g(x)$ dem Produkt der entsprechenden Punkte von
  $f(x)$ und $g(x)$ entspricht, führt eine kleine Vergrösserung von $a$ durch eine winzig kleine Zahl $h$ zu einer Vergrösserung von $f(a)g(a)$ durch die Summe von $f(a)g'(a)h$ und
  $f'(a)hg(a)$. Also,
\begin{align*}
\frac{\Delta y}{\Delta x} &\approx \frac{f(a)g'(a)h+f'(a)g(a)h + f'(a)g'(a)h^2}{h}\\
&\approx f(a)g'(a) + f'(a)g(a).
\end{align*}}
\end{marginfigure}

\begin{proof}
Wir beginnen mit der Definition der Ableitung:
\[
\ddx (f(x)g(x)) = \lim_{h \to0} \frac{f(x+h)g(x+h) - f(x)g(x)}{h}
\]
Nun verwenden wir einen kleinen Trick; wir addieren $0 = -f(x+h)g(x) + f(x+h)g(x)$:
\begin{align*}
&=\lim_{h \to0} \frac{f(x+h)g(x+h){\color{penColor2}-f(x+h)g(x) + f(x+h)g(x)}- f(x)g(x)}{h} \\ 
&=\lim_{h \to0} \frac{f(x+h)g(x+h)-f(x+h)g(x)}{h} + \lim_{h \to0} \frac{f(x+h)g(x)- f(x)g(x)}{h}.
\end{align*}
Da beide Funktionen $f(x)$ und $g(x)$ differenzierbar sind, sind diese auch stetig, vergleichen Sie dazu Satz~\ref{theorem:diff-cont}. Also
\begin{align*}
&=\lim_{h \to0} f(x+h)\frac{g(x+h)-g(x)}{h} + \lim_{h \to0} \frac{f(x+h)- f(x)}{h}g(x) \\ 
&=\lim_{h \to0} f(x+h)\lim_{h \to0}\frac{g(x+h)-g(x)}{h} + \lim_{h \to0} \frac{f(x+h)- f(x)}{h}\lim_{h \to0}g(x) \\ 
&=f(x)g'(x) + f'(x)g(x).
\end{align*}
\end{proof}



Kehren wir nun zum Beispiel vom Anfang zurück:
\begin{example} 
Sei $f(x)=(x^2+1)$ und $g(x)=(x^3-3x)$. Berechnen Sie:
\[
\ddx f(x)g(x).
\]
\end{example}
\begin{solution}
Schreiben Sie
\begin{align*}
\ddx f(x)g(x) &= f(x)g'(x) + f'(x)g(x)\\
&=(x^2+1)(3x^2-3) + 2x(x^3-3x).
\end{align*}
Wir könnten an diesem Punkt aufhören, wir wissen aber, dass ein Ausmultiplizieren auf unsere Lösung vom Anfang führen sollte, also machen wir dies:
\begin{align*}
(x^2+1)(3x^2-3) + 2x(x^3-3x) &= 3x^4-3x^2 +3x^2 -3 + 2x^4-6x^2\\
&=5x^4-6x^2-3,
\end{align*}
was genau unserer Lösung oben entspricht.
\end{solution}




\begin{exercises}

\noindent Berechnen Sie:

\twocol

\begin{exercise} $\ddx x^3(x^3-5x+10)$
\begin{answer} $3x^2(x^3-5x+10)+x^3(3x^2-5)$
\end{answer}\end{exercise}

\begin{exercise} $\ddx (x^2+5x-3)(x^5-6x^3+3x^2-7x+1)$
\begin{answer} $(x^2+5x-3)(5x^4-18x^2+6x-7)+(2x+5)(x^5-6x^3+3x^2-7x+1)$
\end{answer}\end{exercise}

\begin{exercise} $\ddx e^{2x} = \ddx \left(e^x \cdot e^x\right)$
\begin{answer}
$2e^{2x}$
\end{answer}
\end{exercise}

\begin{exercise} $\ddx e^{3x}$
\begin{answer} $3e^{3x}$
\end{answer}\end{exercise}


\begin{exercise} $\ddx 3x^2e^{4x}$
\begin{answer} $6xe^{4x}+12x^2e^{4x}$
\end{answer}\end{exercise}


\begin{exercise} $\ddx \frac{3e^x}{x^{16}}$
\begin{answer} $\frac{-48e^x}{x^{17}}+\frac{3e^x}{x^{16}}$
\end{answer}\end{exercise}


\endtwocol


\begin{exercise} 
Verwenden Sie die Produktregel, um die Ableitung von $f(x)=(2x-3)^2$ zu berechnen. Skizzieren Sie die Funktion. Finden Sie die Gleichung der Tangente an die Kurve im Punkt $x=2$.  Skizzieren Sie diese Tangente im Punkt $x=2$.
\begin{answer} $f'=4(2x-3)$, $y=4x-7$
\end{answer}\end{exercise}

\noindent Benutzen Sie die folgende Tabelle um die nächsten 4 Aufgaben zu lösen. $\left.\ddx f(x)\right|_{x=a}$ ist die Ableitung der Funktion $f(x)$ am Punkt $x=a$.
\[
\begin{tchart}{lllll}
 x    & 1 & 2  & 3 & 4 \\ \hline 
 f(x) & -2 & -3 & 1 & 4 \\
f'(x) & -1 &  0 & 3 & 5\\
 g(x) &  1 &  4 & 2 & -1 \\
g'(x) &  2 &  -1 & -2 & -3\\
\end{tchart}
\]

\twocol
\begin{exercise} $\left. \ddx f(x)g(x) \right|_{x=2}$
\begin{answer} $3$
\end{answer}\end{exercise}

\begin{exercise} $\left. \ddx xf(x) \right|_{x=3}$
\begin{answer} $10$
\end{answer}\end{exercise}

\begin{exercise} $\left. \ddx xg(x) \right|_{x=4}$
\begin{answer} $-13$
\end{answer}\end{exercise}

\begin{exercise} $\left. \ddx f(x)g(x) \right|_{x=1}$
\begin{answer} $-5$
\end{answer}\end{exercise}
\endtwocol

\end{exercises}





\section{Die Quotientenregel}

\index{Quotientenregel} 

Wir benötigen eine Regel, um
\[
\ddx \frac{f(x)}{g(x)}
\]
mit Hilfe von $f'(x)$ und $g'(x)$ zu berechnen. Einen Teil des Problems haben wir bereits im vorhergehenden Unterkapitel gelöst: $f(x)/g(x)= f(x)\cdot(1/g(x))$, dies ist ein Produkt, also können wir die Ableitung berechnen, sobald wir $f'(x)$ und
$(1/g(x))'$ kennen. Dies bringt uns zur nächsten Ableitungsregel:

\begin{mainTheorem}[Die Quotientenregel]\index{Ableitungsregeln!Quotient}\index{Quotientenregel}\label{theorem:quotient-rule}
Sind $f(x)$ und $g(x)$ differenzierbar, dann ist
\[
\ddx \frac{f(x)}{g(x)} = \frac{f'(x)g(x)-f(x)g'(x)}{g(x)^2}.
\]
\end{mainTheorem}
\begin{proof}
Wissen wir wie man die Ableitung
\[
\ddx \frac{1}{g(x)}
\]
berechnet, dann können wir die Produktregel benutzen, um den Beweis zu führen.
\begin{align*}
\ddx\frac{1}{g(x)}&=\lim_{h\to0} \frac{\frac{1}{g(x+h)}-\frac{1}{g(x)}}{h} \\
&=\lim_{h\to0} \frac{\frac{g(x)-g(x+h)}{g(x+h)g(x)}}{h} \\
&=\lim_{h\to0} \frac{g(x)-g(x+h)}{g(x+h)g(x)h} \\
&=\lim_{h\to0} -\frac{g(x+h)-g(x)}{h} \frac{1}{g(x+h)g(x)} \\
&=-\frac{g'(x)}{g(x)^2}.
\end{align*}
Nun können wir dies zusammen mit der Produktregel verwenden:
\begin{align*}
\ddx\frac{f(x)}{g(x)} &=f(x)\frac{-g'(x)}{g(x)^2}+f'(x)\frac{1}{g(x)}\\
&=\frac{-f(x)g'(x)+f'(x)g(x)}{g(x)^2}\\
&=\frac{f'(x)g(x)-f(x)g'(x)}{g(x)^2}.
\end{align*}

\end{proof}


\begin{example}
Berechnen Sie:
\[
\ddx \frac{x^2+1}{x^3-3x}.
\]
\end{example}

\begin{solution}
Schreiben Sie:
\begin{align*}
\ddx \frac{x^2+1}{x^3-3x} &= \frac{2x(x^3-3x)-(x^2+1)(3x^2-3)}{(x^3-3x)^2}\\
&=\frac{-x^4-6x^2+3}{(x^3-3x)^2}.
\end{align*}
\end{solution}

Es ist oft möglich, Ableitungen auf mehr als eine Art zu berechnen. Da jeder Quotient auch als Produkt dargestellt werden kann, ist es auch möglich, die Produktregel zu verwenden um die Ableitung zu berechnen, dies ist jedoch nicht umbedingt immer einfacher.

\begin{example}
Berechnen Sie 
\[
\ddx \frac{625-x^2}{\sqrt{x}}
\] 
auf zwei Arten. Zunächst mit der Quotientenregel, danach mit der Produktregel.
\end{example}

\begin{solution} 
Zuerst berechnen wir die Ableitung mit der Quotientenregel:
\[
\ddx \frac{625-x^2}{\sqrt{x}} = \frac{\left(-2x\right)\left(\sqrt{x}\right) - (625-x^2)\left(\frac{1}{2}x^{-1/2}\right)}{x}.
\]
Danach mit der Produktregel:
\begin{align*}
\ddx \frac{625-x^2}{\sqrt{x}} &= \ddx \left(625-x^2\right)x^{-1/2}\\
&=\left(625-x^2\right)\left(\frac{-x^{-3/2}}{2}\right)+ (-2x)\left(x^{-1/2}\right).
\end{align*}
Mit etwas Algebra lassen sich beide Lösungen umformen zu:
\[
-\frac{3x^2+625}{2x^{3/2}}.
\]
\end{solution}


\begin{exercises}

\noindent Bestimmen Sie die Ableitungen mit Hilfe der Quotientenregel:

\twocol

\begin{exercise} ${x^3\over x^3-5x+10}$
\begin{answer} ${3x^2\over x^3-5x+10}-{x^3(3x^2-5)\over (x^3-5x+10)^2}$
\end{answer}\end{exercise}

\begin{exercise} ${x^2+5x-3\over x^5-6x^3+3x^2-7x+1}$
\begin{answer} ${2x+5\over x^5-6x^3+3x^2-7x+1}-
{(x^2+5x-3)(5x^4-18x^2+6x-7)\over(x^5-6x^3+3x^2-7x+1)^2}$
\end{answer}\end{exercise}


\begin{exercise} $\frac{e^x-4}{2x}$
\begin{answer} $\frac{2xe^x-(e^x-4)2}{4x^2}$
\end{answer}\end{exercise}

\begin{exercise} $\frac{2-x-\sqrt{x}}{x+2}$
\begin{answer} $\frac{(x+2)(-1-(1/2)x^{-1/2}) - (2-x-\sqrt{x})}{(x+2)^2}$
\end{answer}\end{exercise}
\endtwocol


\begin{exercise} Finden Sie eine Gleichung für die Tangente zu $f(x) = (x^2 -
4)/(5-x)$ im Punkt $x= 3$.  
\begin{answer} $y=17x/4-41/4$ 
\end{answer}\end{exercise}

\begin{exercise}  Finden Sie eine Gleichung für die Tangente zu
$f(x) = (x-2)/(x^3 + 4x - 1)$ im Punkt $x=1$.
\begin{answer} $y=11x/16-15/16$
\end{answer}\end{exercise}

\begin{exercise} Die Kurve $y=1/(1+x^2)$ ist ein Beispiel einer Klasse von Funktionen die ``Versiera der Maria Agnesi''\marginnote{Aufgrund eines Übersetzungsfehlers heisst die Kurve im Englischen ``witch of Agnesi''. Der Grund: Im Italienischen heisst die Kurve la versiera di Agnesi, was ``Die Kurve der Agnesi'' bedeutet. Das wurde vom Cambridge Professor John Colson als ``l'awersiera di Agnesi'' gelesen, wobei ``awersiera'', was ``Frau, die gegen Gott gerichtet ist'' bedeutet, dann als ``Hexe'' (``witch'') interpretiert wurde, und die Fehlübersetzung setzte sich durch.} genannt werden (nach der italienischen Mathematikerin Maria Agnesi). Finden Sie die Tangente zu der Kurve im Punkt $x= 5$.
\begin{answer} $y=19/169-5x/338$
\end{answer}\end{exercise}
 

\noindent Benutzen Sie die folgende Tabelle, um die 4 nächsten Aufgaben zu berechnen.
\[
\begin{tchart}{lllll}
 x    & 1 & 2  & 3 & 4 \\ \hline 
 f(x) & -2 & -3 & 1 & 4 \\
f'(x) & -1 &  0 & 3 & 5\\
 g(x) &  1 &  4 & 2 & -1 \\
g'(x) &  2 &  -1 & -2 & -3\\
\end{tchart}
\]

\twocol
\begin{exercise} $\left. \ddx \frac{f(x)}{g(x)} \right|_{x=2}$
\begin{answer} $-3/16$
\end{answer}\end{exercise}

\begin{exercise} $\left. \ddx \frac{f(x)}{x} \right|_{x=3}$
\begin{answer} $8/9$
\end{answer}\end{exercise}

\begin{exercise} $\left. \ddx \frac{xf(x)}{g(x)} \right|_{x=4}$
\begin{answer} $24$
\end{answer}\end{exercise}

\begin{exercise} $\left. \ddx \frac{f(x)g(x)}{x} \right|_{x=1}$
\begin{answer} $-3$
\end{answer}\end{exercise}
\endtwocol


\begin{exercise} Sei $f'(4) = 5$, $g'(4) = 12$, $f(4)g(4)=2$, und $g(4) = 6$,
berechnen Sie $f(4)$ und $\ddx\frac{f(x)}{g(x)}$ im Punkt 4.
\begin{answer} $f(4) = 1/3, \ddx \frac{f(x)}{g(x)} = 13/18$
\end{answer}\end{exercise}

\end{exercises}
